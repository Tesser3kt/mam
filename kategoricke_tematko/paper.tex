\documentclass[letterpaper,11pt,leqno]{article}
\usepackage{paper}
\bibliographystyle{bibliography}

% Enter paper title to populate the PDF metadata:
\hypersetup{pdftitle={Minimalist LaTeX Template for Academic Papers}}

% Enter BibTeX file with references:
\newcommand{\bib}{bibliography.bib}

% Enter PDF file with figures:
\newcommand{\pdf}{figures.pdf}

\begin{document}

% Enter title:
\title{Strojíme matiku puntíky a šipkami}

% Enter authors:
\author{Adam Klepáč}

% Enter date:
\date{\today}   

% Enter permanent URL (can be commented out):
\available{https://github.com/Tesser3kt/mam}

\begin{titlepage}
\maketitle

% Enter abstract:
V tomto témátku se podíváme na to, co jsou \emph{kategorie} a jak se dají použít
k výtvoru mnoha základních stavebních kamenů matematiky, jako \emph{logiky} nebo
\emph{množin}.

\end{titlepage}

% Enter main text:
\section*{Úvod}
\label{sec:uvod}

Mám rád obrázky. Máš rád obrázky. Má rád obrázky. Máme rádi obrázky. Máte rádi
obrázky. Mají rádi obrázky.

Teď vážně, kdo nemá rád obrázky. Věřili byste nám, drazí čtenáři, tvrdili-li
bychom vám, že celou matematiku umíme překreslit do obrázků? Ovšem, ta sama o
sobě není nikterak odvážná výpověď; vždyť písmena, čísla i vše ostatní na papíře
jsou vlastně, obrázky. My zde nechováme na mysli ledajaké obrázky, nýbrž
obrázky, které vzniknou kresbou pouze dvou věcí -- puntíků ($\bullet$) a šipek
($\longrightarrow$) mezi puntíky. Co však míníme slovy \uv{překreslit celou
matematiku}, zůstane ještě nějakou chvíli oděno v rouchu tajemství.

\section{Teorie kategorií}
\label{sec:teorie-kategorii}

Teorie kategorií je matematická disciplína, jež postupně vznikala v průběhu 20.
století. Za její debut v širém matematickém světe lze považovat knihu
\emph{Categories for the Working Mathematician}~\footnote{V době vzniku této
 knihy byla teorie kategorií považována za abstraktní \uv{hrátky} matematiků,
 již měli dostatek majetku, aby nemuseli pracovat. Užitečnost této teorie i v
principu praktičtějším odvětvím ukázal právě S. Mac Lane -- odtud přívlastek
\emph{working} v názvu.} od Saunderse Mac Lana vydanou roku 1971. Dnes se
vskutku štědře využívá především v oblasti algebry.

Jsouc někdy přezdívána \emph{metamatematikou}, teorie kategorií zkoumá vpravdě
matematiku jako takovou. Je vlastně jakýmsi pohledem seshora na to rašeliniště
struktur a konceptů, které vypadají, že mají mnoho společného, ale i mnoho
rozdílného. Teorie kategorií zkoumá právě ty vlastnosti, které spolu všechny
sdílejí.

V tomto témátku se s ní velmi pravděpodobně setkáte poprvé. Setkání bude to
velmi neformální, intuitivní a obrázkové, protože na víc schází času i smyslu.
Ba, před samotným seznámením se stroze rozhovoříme o jednom zásadním rozumovém
kroku v pojetí matematiky (a částečně světa), jejž je radno pro zdárné pochopení
učinit.

V teorii kategorií nebývá zvykem definovat věci určením toho, \uv{co jsou},
nýbrž výčtem \uv{vlastností, které splňují} či \uv{informací, které nesou}.
Místo aby řekli: \uv{Toto je hrnek,} kategorici dějí spíše: \uv{Toto je věc,
která slouží k uskladnění tekutiny během pití.} A \dots pak dalších mnoho stran
dokazují, že všechny takové věci jsou sobě podobné. Tento velmi svobodný přístup
má podíl na úspěchu, který teorie kategorií zaznamenala. Totiž, logika i teorie
množin (nikoli explicitně, ale svým výkladem) v sobě již přirozeně skýtají
jistou \emph{intepretaci} -- logická spojka $ \wedge $ znamená \uv{a zároveň},
množina je \uv{souhrn prvků, jež spolu souvisejí} a podobně. Interpretace
objektu podvědomě zužuje náš obzor a v zásadě nás nutí s ním pracovat konkrétním
způsobem. Proto vám teď neprozradíme, co kategorie \emph{je} (protože to
nevíme), ale jenom a pouze, \emph{co ji tvoří}. Následující \uv{definice} je
šita tak, aby v~sobě shrnovala jen ty zcela nejvšednější vlastnosti všemožných
matematických struktur (možná jste slyšeli o \emph{grupách}, \emph{okruzích} či
\emph{vektorových prostorech}). To není bez jejich hlubší znalosti snadné
prohlédnout, prosíme pročež shovívavé čtenáře o důvěru.

Nyní, \textbf{kategorie} $\mathcal{A}$ je dána následujícími daty:
\begin{itemize}
 \item \textbf{puntíky} (formálně \textbf{objekty}), které budeme značit
  symbolem $\bullet$ či malými písmeny latinské abecedy ($a,b,c,\ldots$),
  bude-li třeba formálního zápisu;
 \item \textbf{šipky} (formálně \textbf{morfismy}), které vždy vedou z puntíku
  do puntíku. Ty budeme značit šipkou $\longrightarrow$, popřípadě malými
  písmeny řecké abecedy ($\alpha,\beta,\gamma,\ldots$). Formální zápis faktu, že
  šipka $\alpha$ vede z puntíku $a$ do puntíku $b$ vypadá takto:
  \[
   a \overset{\alpha}{\longrightarrow} b \quad \text{nebo} \quad \alpha: a \to
   b.
  \]
  Všechny šipky z $a$ do $b$ v kategorii $\mathcal{A}$ značíme
  $\mathcal{A}(a,b)$.
\end{itemize}

Před dokončením definice kategorie se stručně vyjádříme ke značení. Fakt, že
objekt $a$ patří do kategorie $\mathcal{A}$ se obvykle píše jako $a \in
\mathcal{A}$. To je ovšem pouze drzé zneužití značení z teorie množin! \emph{V
žádném smyslu} neznamená, že $a$ je \emph{prvkem} kategorie $\mathcal{A}$,
protože $\mathcal{A}$ rozhodně nemusí být množina. Ježto si silně nepřejeme,
abyste nebozí čtenáři přemýšleli o kategoriích jako o \emph{množinách} a o
objektech jako o \emph{prvcích}, tohoto tradičního značení se zdržíme a budeme
psát například $a \blacktriangleleft \mathcal{A}$, abychom vyjádřili, že $a$ je
objekt kategorie $\mathcal{A}$. Ještě jednou a znovu -- to neznamená, že $a$
\emph{patří do} či \emph{leží v} $\mathcal{A}$. Onen vztah pouze vyjadřuje, že
objekt $a$ je součástí dat popisujících kategorii $\mathcal{A}$. Podobně, fakt,
že $\alpha$ je šipka z $a$ do $b$ v~kategorii $\mathcal{A}$ budeme někdy psát
jako $\alpha \blacktriangleleft \mathcal{A}(a,b)$.

V teorii kategorií je velmi zásadní pojem \emph{komutativity diagramu}. Také
slovní spojení jistě budí děs a hrůzu: \uv{Neslíbili jste neformální úvod?}
Ovšem, \textbf{diagram} je zcela libovolný obrázek z puntíků a šipek mezi
puntíky, o kterém lze prohlásit, že má nějaký počáteční puntík a koncový puntík
(po směru šipek, pochopitelně). Takový diagram \textbf{komutuje}, když jsou
všechny cesty z počátečního puntíku do koncového puntíku v kategorii
$\mathcal{A}$ stejné.

\begin{example}
 Následující tři diagramy
 \begin{figure}[H]
  \centering
  \begin{subfigure}[b]{.3\textwidth}
   \centering

  \end{subfigure}

 \end{figure}
\end{example}

Navíc,
\begin{itemize}
 \item šipky lze \textbf{skládat}. To znamená, že kdykoli v kategorii
  $\mathcal{A}$ nastává situace
  \begin{figure}[H]
   \centering
   \begin{tikzcd}
    & \bullet \drar &\\
    \bullet \urar & & \bullet,
   \end{tikzcd}
  \end{figure}
  pak existuje šipka, která tento trojúhelník doplňuje. Schematicky (a toto
  značení budeme používat i nadále)
  \begin{figure}[H]
   \centering
   \begin{tikzcd}
    & \bullet \drar &\\
    \bullet \ar[ur] \arrow[rr,"\exists",dashed] & & \bullet.
   \end{tikzcd}
  \end{figure}
  Formálně píšeme, že pro libovolné puntíky $a,b,c$ a šipky $a
  \overset{\alpha}{\longrightarrow} b, b \overset{\beta}{\longrightarrow} c$,
  existuje šipka $a \overset{\alpha \beta}{\longrightarrow} c$, které se říká
  jejich \textbf{složení};
 \item toto skládání je \textbf{asociativní}. Nezáleží na tom, jestli nejdříve
  složíme $\alpha$ s $\beta$ a po $\alpha\beta$ pokračujeme šipkou $\gamma$,
  nebo nejdřív vyrobíme $\beta\gamma$ a před touto jdeme šipkou $\alpha$. To je,
  soudíme, velmi přirozená podmínka. Formálně $(\alpha\beta)\gamma =
  \alpha(\beta\gamma)$, a obrázkově
  \begin{figure}[H]
   \centering
   \begin{subfigure}[h]{.4\textwidth}
    \raggedleft
    \begin{tikzcd}[sep=large]
     & \bullet \dar["\beta"] & \bullet \\
     \bullet \urar["\alpha"] \rar[thick,"\alpha\beta"] & \bullet \urar[thick,"\gamma"'] &
    \end{tikzcd}
   \end{subfigure}
   \begin{subfigure}[h]{.1\textwidth}
    \centering
    {\huge $=$}
   \end{subfigure}
   \begin{subfigure}[h]{.4\textwidth}
    \raggedright
    \begin{tikzcd}[sep=large]
     & \bullet \dar["\beta"] \rar[thick,"\beta\gamma"'] & \bullet; \\
     \bullet \urar[thick,"\alpha"] & \bullet \urar["\gamma"'] &
    \end{tikzcd}
   \end{subfigure}
  \end{figure}
 \item každý puntík má kolem sebe jednu speciální \textbf{smyčku} -- šipku, jež
  v něm začíná i končí. Speciální v tom smyslu, že její složení s libovolnou
  jinou šipkou nic nedělá. Je to vlastně taková \uv{zůstaň na místě} šipka.
  Formálně, ke každému puntíku $a$ existuje šipka $a
  \overset{1_a}{\longrightarrow} a$ taková, že kdykoli vezmeme šipku $\alpha$,
  která začíná v $a$, pak $1_a \alpha = \alpha$, a kdykoli vezmeme šipku
  $\beta$, která končí v $a$, pak $\beta 1_a = \beta$. Obrázkově
  \begin{figure}[H]
   \centering
   \begin{subfigure}[b]{.4\textwidth}
    \raggedleft
    \begin{tikzcd}[sep=large]
     \bullet \rar["\alpha"] \ar[loop,in=225,out=135,looseness=5,"1_a"',shorten
     >=-5pt] & \bullet
    \end{tikzcd}
   \end{subfigure}
   \begin{subfigure}[H]{.1\textwidth}
    \centering
    \vspace{3pt}
    {\huge $=$}
   \end{subfigure}
   \begin{subfigure}[b]{.4\textwidth}
    \raggedright
    \begin{tikzcd}[sep=large]
     \bullet \rar["\alpha"] & \bullet
    \end{tikzcd}
   \end{subfigure}
  \end{figure}
  \vspace*{-3em}
  \begin{figure}[H]
   \centering
   \begin{subfigure}[b]{.4\textwidth}
    \raggedleft
    \begin{tikzcd}[sep=large]
     \bullet \ar[loop,in=225,out=135,looseness=5,"1_a"',shorten
     >=-5pt] & \bullet \lar["\beta"]
    \end{tikzcd}
   \end{subfigure}
   \begin{subfigure}[H]{.1\textwidth}
    \centering
    \vspace{3pt}
    {\huge $=$}
   \end{subfigure}
   \begin{subfigure}[b]{.4\textwidth}
    \raggedright
    \begin{tikzcd}[sep=large]
     \bullet & \bullet \lar["\beta"]
    \end{tikzcd}
   \end{subfigure}
  \end{figure}
\end{itemize}

\begin{example}
 \label{exam:categories}
 Všechny tyto tři obrázky představují kategorie.
 \begin{figure}[H]
  \centering
  \begin{subfigure}[b]{.3\textwidth}
   \centering
   \begin{tikzcd}
    \bullet \rar[shift left=.5ex, "\alpha"]
    \ar[loop,in=225,out=135,looseness=5,shorten >=-5pt,"1_a"'] &
    \bullet \lar[shift left=.5ex, "\beta"]
    \ar[loop,in=315,out=45,looseness=5,shorten >=-5pt, "1_b"]
   \end{tikzcd}
  \end{subfigure}
  \begin{subfigure}[b]{.3\textwidth}
   \centering
   \begin{tikzcd}
    \bullet \ar[loop,out=0,in=90,looseness=7, "1_a"']
    \ar[loop,out=120,in=210,looseness=5,shorten >=-5pt, "\alpha"']
    \ar[loop,out=240,in=330,looseness=5,shorten >=-5pt, "\beta"']
   \end{tikzcd}
  \end{subfigure}
  \begin{subfigure}[b]{.3\textwidth}
   \centering
   \begin{tikzcd}
    \bullet \ar[r] \ar[d] \ar[dr] \ar[loop,out=90,in=180,looseness=7,shorten
    >=-3pt] & \bullet \ar[d] \ar[loop,out=90,in=0,looseness=7,shorten
    >=-3pt]\\
    \bullet \ar[r] \ar[loop,out=270,in=180,looseness=7,shorten
    >=-3pt] & \bullet \ar[loop,out=270,in=0,looseness=7,shorten
    >=-3pt]
   \end{tikzcd}
  \end{subfigure}
 \end{figure}


 Stojí za zmínku, že poslední obrázek je kategorií jedině v případě, když
 složení šipek po obvodu dolního trojúhelníku je stejné jako složení šipek po
 obvodu horního trojúhelníku. Tedy jedině, když se přerušované šipky na
 následujících obrázcích rovnají.
 \begin{figure}[H]
  \centering
  \begin{subfigure}[b]{.4\textwidth}
   \raggedleft
   \begin{tikzcd}
    \bullet \ar[d,"\alpha"'] \ar[dr,dashed,"\alpha\beta"] \\
    \bullet \ar[r,"\beta"'] & \bullet 
   \end{tikzcd}
  \end{subfigure}
  \begin{subfigure}[H]{.1\textwidth}
   \centering
   \vspace{3pt}
   {\huge $=$}
  \end{subfigure}
  \begin{subfigure}[b]{.4\textwidth}
   \raggedright
   \begin{tikzcd}
    \bullet \ar[r,"\gamma"] \ar[dr,dashed,"\gamma\delta"'] & \bullet
    \ar[d,"\delta"] \\
                                                           & \bullet
   \end{tikzcd}
  \end{subfigure}
 \end{figure}
 Jinak bychom museli nakreslit z levého horního puntíku do pravého dolního
 puntíku šipky dvě -- jednu za $\alpha\beta$ a druhou za $\gamma\delta$.
\end{example}

\begin{problem}[0,5 b]
 Rozhodněte, zda následující obrázky \textbf{mohou} být kategoriemi. Čili, lze
 označit některé šipky za stejné (jako na třetím obrázku v
 příkladě~\ref{exam:categories}) a splnit tak všechny podmínky definice
 kategorie?
 \begin{figure}[H]
  \centering
  \begin{subfigure}[b]{.3\textwidth}
   \centering
   \begin{tikzcd}
    \bullet \ar[r] \ar[loop,out=90,in=180,looseness=7,shorten >=-3pt]\ar[dr] &
    \bullet \ar[d] \ar[d] \ar[loop,out=90,in=0,looseness=7,shorten
    >=-3pt]\\
    \bullet \ar[u] \ar[loop,out=270,in=180,looseness=7,shorten
    >=-3pt] \ar[r] & \bullet \ar[loop,out=270,in=0,looseness=7,shorten
    >=-3pt]
   \end{tikzcd}
  \end{subfigure}
  \begin{subfigure}[b]{.3\textwidth}
   \centering
   \begin{tikzcd}

   \end{tikzcd}
  \end{subfigure}
  \begin{subfigure}[b]{.3\textwidth}
   \centering
   \begin{tikzcd}
    \bullet \ar[r] \ar[d] \ar[dr] \ar[loop,out=90,in=180,looseness=7,shorten
    >=-3pt] & \bullet \ar[d] \ar[loop,out=90,in=0,looseness=7,shorten
    >=-3pt]\\
    \bullet \ar[r] \ar[loop,out=270,in=180,looseness=7,shorten
    >=-3pt] & \bullet \ar[loop,out=270,in=0,looseness=7,shorten
    >=-3pt]
   \end{tikzcd}
  \end{subfigure}
 \end{figure}
\end{problem}
\end{document}
